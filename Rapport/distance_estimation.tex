

Pour l'estimation des distance on doit décomposer l'espace en plusieurs dimensions. Pour simplifier on va diviser l'espace en 2 dimensions.

Pour 2 têtes, On prend comme référence le plan qui passe par ces 2 têtes et dont une des 2 dimensions est parrallêle au sol (donc au bord inférieur de l'image). Ainsi on a les 2 dimensions:

\begin{itemize}
    \item la dimension X est la dimension parrallêle au bord inférieur de l'image
    \item la dimension Y est celle perpendiculaire à celle-ci, mais qui suit le plan passant par les 2 têtes.
\end{itemize}


Ainsi, en obtenant la projection de lu vecteur disctantce sur ces 2 dimensions, on calcul la distance totale avec le théorème de pythagore.

Par ailleurs on notera : $(x_1, y_1)$ et $(x_2, y_2) $les coordonnées des 2 têtes (en pixel), $(d_1,d_2)$ leur profondeur respective (en mêtre) et $(dX, dY)$ le vecteur distance entre les 2 têtes (en mêtres). Et $f$ la focale de la caméra (en pixel).

\subsection{distance sur l'axe Y}

Pour calculer la distance sur l'axe y, qui est en fait la différence de profondeur corrigé pour prendre en compte la position relative des personnes, on utilise le théorème d'alkachy.

\subsection{distance sur l'axe X}

Pour la calculer la distance dans cette dimensions c'est assez facile, en utilisant la formume \ref{eq:camera_relation} on détermine la distance en mêtre au niveau de x1 avec d1, puis au niveau de x2 avec d2.
On fait l'hypothèse que la perspecive est parfaitement plongeante ce qui fait qu'il faut enlever la moitié de la différence pour obtenir la distance entre les 2 têtes (avec $d_2>d_1$).

\begin{align}
    dX1 &= (x_2-x_1) * d1 / focal \\
    dX2 &= (x_2-x_1) * d2 / focal \\
    dX  &= dX2 - (dX2-dX1)/2 \\
        &= (dX2 + dX1) /2
\end{align}

