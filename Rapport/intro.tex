 L’analyse des foules à partir d’images constitue un domaine de recherche en pleine expansion, avec des applications variées allant de la gestion de la sécurité publique à l’optimisation des flux dans les espaces urbains. Parmi les défis inhérents à cette tâche, l’estimation précise des distances entre les individus dans une foule à partir d’une seule image monoculaire reste particulièrement complexe. En effet, l’absence de données tridimensionnelles directes et la variabilité des scènes rendent cette problématique difficile à résoudre avec les approches classiques. 

Notre volonté pour réaliser cette tâche proviens de notre projet de mentions, qui vise à détecter les dangers dans une foules à partir des données de déplacement des individus. Cependant il n'existe pas vraiment de set de données contenant le déplacement des personnes au sein d'une foule, il est cependant facile de se procurer des vidéos de foules. Notre meta objectif est donc de pouvoir extraire les données de déplacements des individus d'une foule à partir d'une simple vidéo. Il était donc nécessaire dans la construction de notre solution de penser à l'adaptabilité future à un flux vidéo.

Dans ce contexte, notre projet s’est attaché à développer une méthode innovante pour estimer les distances interpersonnelles sur une image (dans un premier temps), en se concentrant spécifiquement sur les têtes, qui représentent les éléments les plus distinctement détectables dans une foule. Face à l’absence de solutions préexistantes dans la littérature, nous avons conçu une approche hybride combinant des techniques de vision par ordinateur et des principes d’optique géométrique. Ce rapport présente notre méthodologie, qui repose sur l’intégration d’un modèle de détection d’objets affiné et d’un modèle d’estimation de profondeur, ainsi que les premiers résultats obtenus. En explorant cette problématique, notre objectif est de poser les bases d’une solution robuste et généralisable, tout en identifiant les axes d’amélioration.